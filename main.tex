% -------------------------------
% == REPORT 2 ==
% -------------------------------
% -------------------------------
% Madison: Odd numbered sections
% Jake: Even numbered sections
% -------------------------------
% -------------------------------
%

\documentclass[conf]{new-aiaa}

\usepackage[utf8]{inputenc}
\usepackage{float}
\usepackage{graphicx}
\usepackage{amsmath}
\usepackage[version=4]{mhchem}
\usepackage{siunitx}
\usepackage{longtable,tabularx}
\setlength\LTleft{0pt} 
\usepackage{hyperref}
\usepackage{nomencl}
\usepackage{lscape}

\linespread{2}  

\title{Design Report 03: Twin Sea Lion}

\author{
    Madison Junker\footnote{Student ID: 102736535 }, Jacob Killelea\footnote{Student ID: 10550162 } \\
    \emph{ASEN 4138, University of Colorado Boulder, November 16, 2018}
}

\begin{document}

\clearpage
\maketitle
\thispagestyle{empty}

\newpage
\pagenumbering{roman}
\tableofcontents
\addcontentsline{toc}{section}{\listfigurename}
\listoffigures
\addcontentsline{toc}{section}{\listtablename}
\listoftables
\newpage
\printnomenclature[25mm]

\section*{Nomenclature}

{\renewcommand\arraystretch{1.0}
\noindent\begin{longtable*}{@{}l @{\quad=\quad} l@{}}
$AAA$     	               & Advanced Aircraft Analysis Program \\
$AR_W$    	               & Aspect Ratio                       \\
$b_W$	  	               & Wing Span                          \\
$\bar{c}_W$	  	           & Mean Geometric Chord               \\
$i_W$		               & Incidence Angle                    \\
KTAS                       & Knots True Airspeed                \\
$l_f$     	               & Length of fuselage                 \\
$MSL$     	               & Mean Sea Level Altitude            \\
$S_W$     	               & Wing Area                          \\
$TWR$     	               & Thrust to weight ratio             \\
$\epsilon_W$               & Wing Twist Angle                   \\
$\Lambda_{c/4w}$[$^\circ$] & Wing Sweep Angle                   \\
$\lambda_W$	               & Taper Ratio                        \\
$\lambda_{c/4w}$           & Quarter-chord Sweep Angle          \\
$\Gamma_W$	               & Dihedral                           \\
\end{longtable*}}

\newpage
\pagenumbering{arabic}

% 1
\section{Introduction}

\section{Preliminary Weight and Balance Analysis}
\subsection{Preliminary Weight Breakdown}
%Define/show axis system
%Explain the empty weight breakdown of your aircraft, including individual tables with structural, powerplant, and fixed equipment weight breakdowns and the x-,y-, and z-cg locations of the various weight components. You can "cut and paste" the tables from your AAA printouts if they are presented in the text as numbered, titled tables. You don't need to repeat them in your appendix
Since the engine weigh 1500 lbs each and our initial weight budget for power plant was 4655.1 lbs, some extra weight for the engines had to be found.
%Explain if you adjusted any weights calculated from the weight fraction table (like the powerplant weight) Give calculations for your lavatory and galley weights, if applicable, and any other detailed weights such as conference tables. Discuss the general method of how you determined component cg locations and how you divided up calculated weights (like tail and gear weights).
From the gerren equation (Pres 18), the propellers have a diameter of 9.25 ft. From AAA, these propellers will weigh about 1800 lbs. With 3000 lbs of engine and about 700 lbs of extra power plant weight (reduced due to electronic actuators).

Power plant weight was determined by dividing the allowed weight in two and subtracting the 901.8 allotted for each side of propeller.

Vertical tail weight was assumed to be 3/8 of allotted tail weight to conform to DHC-6 proportions.

%If you are storing fuel in the fuselage, include a drawing of your fuel tank, including its estimated cg point. Don't forget to include its location in the fuselage for weight and balance calculations

\subsection{Preliminary Weight and Balance Calculation}
%Explain the fully-loaded weight breakdown of your aircraft, including a table with the loaded component weights and x-,y- and z-cg locations. Again, you can "cut and paste" as above.
%Include the loading scenario. Explain how you grouped things (like passengers, munitions, etc.) and determined the order in which items are loaded/unloaded. Show the X-cg and Z-cg excursion plots.
%List the fully-loaded, forward and aft cg locations (need X and Z values for each). Calculated your X-cg shift from most forward to most aft location, and divide by your mean aerodynamic chord. Compare you values (in inches as well as divided by c) with those in Tables 10.3 of Roskam, Part II (posted on Canvas under Technical Documentation).


\section{Empennage Layout Design}
%Present what kind of tail (T-tail, cruciform, or low-mounted) was chosen and why. Also, if any fuel is being stored in the empennage, include a fuel volume calculation
\subsection{Sizing the Horizontal Stabilizer} %AND/OR canard
%Discuss the location and design of the horizontal stabilizer and/or the canard. Need to include: S, b, AR, Gamma c/4, Lambda, lambda, t/c, and airfoil selected for the horizontal tail and/or canard. Calculate and compare values of Vh and/or Vc to those of similar aircraft
%Include AAA printouts of volume coefficient calculations in the appendix and a drawing of the horizontal stabilizer and/or canard in the chapter itself
%Perform a critical Mach number check for each, if applicable. Mcr plots are found in Canvas. Show where your point lies on the plot
\subsection{Sizing the Vertical Stabilizer}
%Discuss the location and design of the vertical stabilizer. Need to include: S, b, AR, Gamma c/4, lambda, t/c, and airfoil selected. Calculate and compare values of Vv to those of similar aircraft.
%Include AAA printouts of volume coefficient calculations in the appendix and a drawing of the vertical stabilizer in the chapter itself
%Perform a Mach critical check, if applicable. Again Mcr plots are found in Canvas. Show where your point lies on the plot
%Go to the stability and control module and calculate the static margin, setting eta hp,off =1. This will give you the static margin of your airplane, which is listed as the output parameter, SM. Comment on the results - is your airplane likely to be longitudinally stable? Unstable? If changes need to be made to either the CG (due to weight and balance) and/or the AC (due to wing/tail location), now is the time to make changes.

\section{Control Surface Layout Design}
\subsection{Sizing the Lateral Control Surfaces}
%Include the AAA printout of aileron calculations in the appendix and a drawing of the wing/aileron in the chapter itself
%Include ca/cw, eta ai and eta ao dimensions, and show that the ailerons does not interfere with the flap location. Include spoiler sizing data here if spoilers are needed. Include (Sa/Sw) calculated from your AAA values and compare it with similar aircraft. Comment - based on your valued, do you expect problems with sufficient roll authority?
\subsection{Sizing the Longitudinal Control Surfaces}
%Include AAA printout of elevator and/or canardvator calculations in the appendix and a drawing of the tail/elevator and/or canard/canardvator in the chapter itself
%Include ce/ch, eta eo dimensions (and/or ccv/cc, eta cvi, and eta cvo dimensions). Include (Se/Sh), and/or (Scv/Sc) calcuated from your AAA values and compare them with similar aircraft. Comment - based on your values, do you expect problems with sufficient pitch authority?
%If you have a movable stabilizer with no elevator and/or a movable canard with no canardvator, state and discuss that in this section
\subsection{Sizing the Directional Control Surfaces}
%Include AAA printout of rudder calculations in the appendix and a drawing of the tail/rudder in the chapter itself
%Include cr/cv, eta ti and eta p0 dimensions. Include (Sr/Sv) calculated from your AAA values and compare it with similar aircraft. Comment - based on your values, do you expect problems with sufficient yaw authority?

\section{Landing Gear Layout Design}
\subsection{Sizing of the Landing Gear}
%Describe how you chose the landing gear configuration (tricycle, taildragger, etc) for your aircraft and why.
% Include
	%Fixed or retractable
    %Number of struts ("bogeys")
    %Number of tires per strut and selection of tires (diameter and width)
    %Arrangement of tires (dual, tandem, etc)
    %Maximum static load per bogey (Pn and Pm)
    %Weight distribution % - is it within acceptable bounds?
\subsection{Location of the Landing Gear}
%Describe how you chose the location of the gear for your aircraft
%Show that you meet longitudinal and lateral ground clearance requirements
%Show that you meet longitudinal and lateral tip-over requirements
%Include strut diameters and lengths

\subsection{Gear Retraction Volume}
%If application, discuss your gear retraction volume calculations (with growth factors) for all gear and whether the gear will retract into the wings, into the fuselage, or a bubble fairing on the side of the fuselage. Show your calculations, both for what volume is required for retraction and for what is available in the wing, fuselage, or fairing.

\section{Conclusions and Recommendations}
\subsection{Conclusions}

\subsection{Recommendations}

\begin{thebibliography}{99}


\bibitem{pres13} Gerren, "Presentation 13", \url{https://canvas.colorado.edu}

\end{thebibliography}

% 8
\section{Appendix}



\subsection{AAA: Preliminary Weight and Balance Analysis}

\subsection{AAA: Empennage Layout Design}

\subsection{AAA: Control Surface Layout Design}

\subsection{AAA: Landing Gear Layout Design}


\end{document}

